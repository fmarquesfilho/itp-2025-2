\documentclass[10pt]{beamer}
\usepackage[utf8]{inputenc}
\usepackage[T1]{fontenc}
\usepackage[brazilian]{babel}
\usepackage{graphicx}
\usepackage{listings}
\usepackage{amssymb}
\usepackage{amsmath}

\usetheme{Madrid}
\title{Introdução a Técnicas de Programação (2025.2)}
\subtitle{Plano de Curso e Orientações para o Projeto}
\author{Prof. Fernando Figueira}
\institute{DIMAp - UFRN}
\date{22 de Agosto de 2025}

% Configuração do listings para evitar problemas de codificação
\lstset{
    basicstyle=\small\ttfamily,
    breaklines=true,
    frame=single,
    inputencoding=utf8,
    extendedchars=true
}

\begin{document}

\frame{\titlepage}

\begin{frame}{Agenda da Aula}
	\tableofcontents[pausesections]
\end{frame}

\section{Informações Gerais do Curso}

\begin{frame}{Informações Gerais}
	\begin{block}{Período Letivo}
		\begin{itemize}
			\item \textbf{Duração:} 22/08/2025 a 15/12/2025
			\item \textbf{Encontros síncronos:} Segundas, Quartas e Sextas
			\item \textbf{Horário:} T56 (16:40 às 18:20)
			\item \textbf{Modalidade:} Remota
			\item \textbf{Acesso:} Google Meet via SIGAA
		\end{itemize}
	\end{block}
	
	\begin{alertblock}{Importante}
		Aulas de conteúdo novo às quartas-feira (gravadas)\\
		Segundas e sextas para revisão e dúvidas
	\end{alertblock}
\end{frame}

\section{Metodologia e Avaliação}

\begin{frame}{Como Funciona a Avaliação?}
	\begin{columns}
		\begin{column}{0.6\textwidth}
			\textbf{Composição da Nota:}
			\begin{itemize}
				\item \textcolor{blue}{\textbf{Listas de exercícios: 20\%}}
				\item \textcolor{red}{\textbf{Projeto individual: 80\%}}
			\end{itemize}
		\end{column}
		\begin{column}{0.4\textwidth}
			\begin{center}
				\textbf{3 Unidades}\\
				\bigskip
				U1: 22/08 - 30/09\\
				U2: 01/10 - 27/10\\
				U3: 29/10 - 28/11\\
				\bigskip
				\textcolor{red}{Entrega Final: 05/12}
			\end{center}
		\end{column}
	\end{columns}
\end{frame}

\begin{frame}{O que Você Precisa Entregar?}
	\begin{block}{Para cada unidade, você entrega:}
		\begin{enumerate}
			\item \textbf{Link do repositório Git público} (GitHub/GitLab/BitBucket)
			\item \textbf{Vídeo de 5-8 minutos} demonstrando o projeto
			\item \textbf{Relatório técnico} em PDF (3-5 páginas)
		\end{enumerate}
	\end{block}
	
	\begin{alertblock}{Atenção!}
		Apenas o commit mais recente até \textbf{23:59} será corrigido.\\
		Commits frequentes e descritivos são valorizados!
	\end{alertblock}
\end{frame}

\section{Estrutura do Projeto}

\begin{frame}{Evolução do Projeto por Unidade}
	\begin{block}{Unidade 1 (22/8 até 30/09)}
		Variáveis, operadores, vetores, condicionais, repetição, funções básicas
	\end{block}
	
	\begin{block}{Unidade 2 (1/10 até 27/10)}
		Tudo da U1 + strings, estruturas aninhadas, matrizes, ponteiros básicos
	\end{block}
	
	\begin{block}{Unidade 3 (29/10 até 28/11)}
		Tudo anterior + ponteiros avançados, arquivos, registros, modularização
	\end{block}
	
	\begin{exampleblock}{Projeto Final (até 05/12)}
		Versão completa com todos os conceitos integrados
	\end{exampleblock}
\end{frame}

\begin{frame}[fragile]{Estrutura do Repositório}
	\begin{exampleblock}{Sugestão de organização}
		\begin{lstlisting}
nome-do-aluno-itp-2025-2/
|-- projeto/
|   |-- src/             # Codigo-fonte
|   |-- include/         # Cabecalhos
|   |-- Makefile         # Compilacao
|   `-- README.md        # Instrucoes
|-- listas/
|   |-- lista1/
|   |-- lista2/
|   `-- ...
`-- README.md            # Descricao geral
		\end{lstlisting}
	\end{exampleblock}
\end{frame}

\section{Critérios de Avaliação}

\begin{frame}{Como Seu Projeto Será Avaliado?}
	\begin{columns}
		\begin{column}{0.5\textwidth}
			\textbf{Código (40\%):}
			\begin{itemize}
				\item Qualidade e organização (10\%)
				\item Funcionalidade (15\%)
				\item Aplicação dos conceitos (10\%)
				\item Histórico de commits (5\%)
			\end{itemize}
		\end{column}
		\begin{column}{0.5\textwidth}
			\textbf{Relatório (30\%):}
			\begin{itemize}
				\item Clareza e coerência (10\%)
				\item Profundidade técnica (10\%)
				\item Respostas orientadoras (10\%)
			\end{itemize}
			
			\bigskip
			\textbf{Vídeo (30\%):}
			\begin{itemize}
				\item Demonstração (15\%)
				\item Qualidade da explicação (10\%)
				\item Tempo adequado (5\%)
			\end{itemize}
		\end{column}
	\end{columns}
\end{frame}

\begin{frame}{Estrutura do Relatório Técnico}
	\begin{enumerate}
		\item \textbf{Introdução} - Contexto e objetivos
		\item \textbf{Metodologia} - Ferramentas e abordagem
		\item \textbf{Análise do Código} - Principais estruturas
		\item \textbf{Dificuldades e Soluções} - Desafios enfrentados
		\item \textbf{Conclusão} - Aprendizados e melhorias
	\end{enumerate}
	
	\begin{block}{Perguntas que você deve responder:}
		\begin{itemize}
			\item Quais conceitos da unidade foram aplicados?
			\item Como a organização facilita a manutenção?
			\item Quais foram os principais desafios técnicos?
		\end{itemize}
	\end{block}
\end{frame}

\section{Sugestões de Projetos}

\begin{frame}{20 Ideias de Projetos}
	\begin{columns}
		\begin{column}{0.5\textwidth}
			\begin{enumerate}
				\item Sistema de Biblioteca
				\item Jogo da Velha com IA
				\item Calculadora Científica
				\item Gerenciador de Tarefas
				\item Simulador de Banco
				\item Conversor de Unidades
				\item Jogo de Campo Minado
				\item Sistema de Cadastro
				\item Agenda de Contatos
				\item Shell Básico
			\end{enumerate}
		\end{column}
		\begin{column}{0.5\textwidth}
			\begin{enumerate}
				\setcounter{enumi}{10}
				\item Compressão de Texto
				\item Quebra-Cabeça 8 Peças
				\item Sistema de Vendas
				\item Jogo da Vida
				\item Calculadora de Matrizes
				\item Reserva de Passagens
				\item Jogo de Xadrez
				\item Analisador Léxico
				\item Sistema de Logs
				\item Gerenciador Financeiro
			\end{enumerate}
		\end{column}
	\end{columns}
\end{frame}

\begin{frame}{Requisitos para Projetos Próprios}
	\begin{alertblock}{Se você quiser fazer um projeto diferente:}
		\begin{itemize}
			\item Deve usar $\geq$70\% dos tópicos da unidade
			\item Ser desenvolvido em \textbf{linguagem C}
			\item Ter interface de \textbf{linha de comando (CLI)}
			\item Ser \textbf{original} (não copiado)
			\item Ter \textbf{complexidade média/alta}
		\end{itemize}
	\end{alertblock}
	
	\begin{block}{Dica}
		Escolha algo que te motive! Você vai trabalhar nele por 4 meses.
	\end{block}
\end{frame}

\section{Cronograma e Datas}

\begin{frame}{Cronograma Resumido}
	\begin{block}{Semanas 1-6: Unidade 1}
		Variáveis $\rightarrow$ Condicionais $\rightarrow$ Loops $\rightarrow$ Funções $\rightarrow$ Vetores
	\end{block}
	
	\begin{block}{Semanas 7-10: Unidade 2}  
		Strings $\rightarrow$ Loops Aninhados $\rightarrow$ Matrizes $\rightarrow$ Ponteiros Básicos
	\end{block}
	
	\begin{block}{Semanas 11-15: Unidade 3}
		Ponteiros Avançados $\rightarrow$ Funções Complexas $\rightarrow$ Arquivos $\rightarrow$ Registros $\rightarrow$ Modularização
	\end{block}
	
	\begin{exampleblock}{Semana 16: Projeto Final}
		Entrega final e encerramento
	\end{exampleblock}
\end{frame}

\begin{frame}{Entregas Importantes - Não Esqueça!}
	\begin{center}
		\Large
		\textcolor{red}{\textbf{U1: 30/09 até 23:59}}\\[0.5cm]
		\textcolor{red}{\textbf{U2: 27/10 até 23:59}}\\[0.5cm]
		\textcolor{red}{\textbf{U3: 05/12 até 23:59}}\\[0.5cm]
	\end{center}
	
	\bigskip
	\begin{block}{Feriado que afeta o curso}
		03/10/25 - Dia dos Mártires (sexta-feira) - Não haverá aula
	\end{block}
\end{frame}

\section{Motivação e Expectativas}

\begin{frame}{Por Que Este Curso É Importante?}
	\begin{block}{Você vai aprender:}
		\begin{itemize}
			\item \textbf{Fundamentos sólidos} da programação
			\item \textbf{Pensamento lógico} e resolução de problemas
			\item \textbf{Boas práticas} de desenvolvimento
			\item \textbf{Trabalho com projetos reais}
		\end{itemize}
	\end{block}
	
	\begin{exampleblock}{Ao final do curso você será capaz de:}
		\begin{itemize}
			\item Desenvolver programas completos em C
			\item Organizar código de forma profissional
			\item Documentar e apresentar seus projetos
			\item Trabalhar com controle de versão (Git)
		\end{itemize}
	\end{exampleblock}
\end{frame}

\begin{frame}{Dicas para o Sucesso}
	\begin{columns}
		\begin{column}{0.5\textwidth}
			\textbf{Estudos:}
			\begin{itemize}
				\item Pratique regularmente
				\item Não deixe acumular
				\item Participe das aulas
				\item Faça as listas
			\end{itemize}
		\end{column}
		\begin{column}{0.5\textwidth}
			\textbf{Projeto:}
			\begin{itemize}
				\item Comece cedo
				\item Commits frequentes
				\item Documente bem
				\item Teste sempre
			\end{itemize}
		\end{column}
	\end{columns}
	
	\bigskip
	\begin{center}
		\textcolor{blue}{\textbf{Lembre-se: A programação é uma habilidade prática!}}
	\end{center}
\end{frame}

\begin{frame}{Recursos de Apoio}
	\begin{block}{Como tirar dúvidas?}
		\begin{itemize}
			\item \textbf{Aulas síncronas} (segundas e sextas (quarta será conteúdo novo))
			\item \textbf{Discord} - mensagens e fóruns
			\item \textbf{E-mail:} fernando@dimap.ufrn.br
		\end{itemize}
	\end{block}
	
	\begin{exampleblock}{Materiais disponíveis:}
		\begin{itemize}
			\item Gravações das aulas teóricas
			\item Listas de exercícios
			\item Material de apoio no SIGAA
			\item Exemplos de código no repositório do curso: \url{https://github.com/fmarquesfilho/itp-2025-2}
		\end{itemize}
	\end{exampleblock}
\end{frame}

\section{Próximos Passos}

\begin{frame}{O Que Fazer Agora?}
	\begin{enumerate}
		\item \textbf{Preparar o ambiente de desenvolvimento}
		\begin{itemize}
			\item Instalar compilador C (GCC)
			\item Escolher um editor (VS Code, Dev-C++, etc.)
			\item Criar conta no GitHub/GitLab (se já não a tiver)
		\end{itemize}
		
		\item \textbf{Começar a pensar no projeto}
		\begin{itemize}
			\item Revisar as sugestões
			\item Escolher algo que te interesse
			\item Começar simples, evoluir gradualmente
		\end{itemize}
		
		\item \textbf{Acompanhar o cronograma}
		\begin{itemize}
			\item Próxima aula: Preparação do ambiente
			\item Primeira lista em breve
		\end{itemize}
	\end{enumerate}
\end{frame}

\begin{frame}{Expectativas Realistas}
	\begin{alertblock}{É normal...}
		\begin{itemize}
			\item Sentir dificuldade no início
			\item Cometer erros de sintaxe
			\item Precisar de várias tentativas
			\item Ter bugs no código
		\end{itemize}
	\end{alertblock}
	
	\begin{block}{O importante é...}
		\begin{itemize}
			\item \textbf{Persistir} e praticar
			\item \textbf{Aprender} com os erros
			\item \textbf{Buscar ajuda} quando necessário
			\item \textbf{Celebrar} os pequenos sucessos
		\end{itemize}
	\end{block}
\end{frame}

\begin{frame}{Dúvidas?}
	\begin{center}
		{\Large Alguma pergunta sobre:}\\
		\bigskip
		• Metodologia do curso?\\
		• Avaliação?\\
		• Projetos?\\
		• Cronograma?\\
		\bigskip
		\bigskip
		{\Large \textcolor{blue}{Vamos esclarecer tudo agora!}}
	\end{center}
\end{frame}

\begin{frame}
	\begin{center}
		{\Huge Obrigado!}\\
		\bigskip
		{\Large Nos vemos na próxima aula!}\\
		\bigskip
		\bigskip
		{\large fernando@dimap.ufrn.br}\\
		{\large DIMAp - UFRN}
	\end{center}
\end{frame}

\end{document}
