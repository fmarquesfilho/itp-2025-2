% Copyright 2007 by Till Tantau
%
% This file may be distributed and/or modified
%
% 1. under the LaTeX Project Public License and/or
% 2. under the GNU Public License.
%
% See the file doc/licenses/LICENSE for more details.

\documentclass[portuguese,10pt,xcolor=table]{bredelebeamer}
\setbeameroption{show notes}

\usepackage[brazil]{babel}
\usepackage[utf8]{inputenc}
\usepackage{times}
\usepackage{varwidth}
\usepackage{listings} % Código de programas
\usepackage{tikz}
\usepackage[tikz]{bclogo}
\usepackage{tikzsymbols}
\usepackage{pifont}
\usetikzlibrary{arrows,shapes}

\usetikzlibrary{calc,decorations.pathmorphing,patterns}
\pgfdeclaredecoration{penciline}{initial}{
\state{initial}[width=+\pgfdecoratedinputsegmentremainingdistance,
auto corner on length=1mm,]{
\pgfpathcurveto%
{% From
\pgfqpoint{\pgfdecoratedinputsegmentremainingdistance}
{\pgfdecorationsegmentamplitude}
}
{%  Control 1
\pgfmathrand
\pgfpointadd{\pgfqpoint{\pgfdecoratedinputsegmentremainingdistance}{0pt}}
{\pgfqpoint{-\pgfdecorationsegmentaspect
\pgfdecoratedinputsegmentremainingdistance}%
{\pgfmathresult\pgfdecorationsegmentamplitude}
}
}
{%TO 
\pgfpointadd{\pgfpointdecoratedinputsegmentlast}{\pgfpoint{1pt}{1pt}}
}
}
\state{final}{}
}



\everymath{\displaystyle}
\tikzstyle{every picture}+=[remember picture,decoration=penciline]
\DeclareTextFontCommand{\textdf}{\bfseries\color{blue!80}}
%\tikzstyle{every node}+=[decorate]
%\tikzstyle{every path}+=[decorate]
%\tikzstyle{na} = [baseline=-.5ex]

\usepackage[T1]{fontenc}

\def\lecturename{IMD0012 - Introdução às técnicas de programação}

\title{\insertlecture}

\author{Prof. Fernando Figueira\\(adaptado do material do Prof. Rafael Beserra Gomes)}

\institute{UFRN}

\subject{Vetores}

\lecture[]{Vetores}{}


\date{}

\def\exe[#1]{\color{gray}#1\color{black}}
\def\exp[#1]{\color{gray}<\textit{#1}>\color{black}}
\def\espaco{\color{gray}\hspace{0.2cm}\color{black}}
\def\espaco{\color{blue}␣\color{black}}
\def\inativo[#1]{\color{gray}#1\color{black}}

\definecolor{deepgreen}{rgb}{0,0.5,0}
\lstset{
language=C,
basicstyle=\footnotesize\ttfamily,
	%basicstyle=\scriptsize\ttfamily,
keywordstyle=\footnotesize\bfseries\sffamily,
	%keywordstyle=\scriptsize\bfseries\sffamily,
showstringspaces=false,
numbers=left,
numberstyle=\footnotesize,
stepnumber=1,
numbersep=5pt,
tabsize=4,
	%backgroundcolor=\color{blue!05},
backgroundcolor=\color{gray!35},
showspaces=false,
showtabs=false,
stringstyle=\ttfamily\color{red!80!brown},
commentstyle=\ttfamily\color{blue!80},
keywordstyle=\bfseries\color{deepgreen},
escapeinside={\%*}{*)}
}
\renewcommand{\lstlistingname}{Código}
\begin{document}

\usebackgroundtemplate{%
\includegraphics[width=\paperwidth,height=\paperheight]{background2}
}
\begin{frame}
	\maketitle
	\begin{center}
		\tiny
		Material compilado em \today.\\
		Licença desta apresentação:\\
		\includegraphics[height=1.0cm]{by-nc-nd.png}\\
		http://creativecommons.org/licenses/
	\end{center}
\end{frame}


\def\GN[#1]{\colorbox{gray!40}{#1}}
\def\RN[#1]{\colorbox{red!40}{#1}}
\def\BN[#1]{\colorbox{blue!40}{#1}}
\def\ON[#1]{\colorbox{orange!40}{#1}}
\def\WN[#1]{\colorbox{white!40}{#1}}


	\section{Vetores}

		\begin{frame}[c]
			\begin{center}
				\structure{\large Vetores}
			\end{center}
		\end{frame} 

		\begin{frame}{\insertsection}
			\begin{itemize}
				\item \textbf{Problema:} armazenar em uma variável mais de um valor do mesmo tipo
				\item Por exemplo:
					\begin{itemize}
						\item uma sequência de números que o usuário digitou (int)
						\item o histórico de valor do dólar (float)
						\item uma frase como sequência de caracteres (char)
					\end{itemize}
				\item Criar inúmeras variáveis é inviável
					\lstinputlisting{variasVariaveis.c}
			\end{itemize}
		\end{frame}	

		\begin{frame}{\insertsection}
			\begin{itemize}
				\item \textdf{Arranjos (array)}: conjunto de elementos identificáveis por um índice
				\item Arranjos unidimensionais: \textdf{vetores}
				\item Arranjos bidimensionais: \textdf{matrizes} (em aula posterior)
			\end{itemize}
		\end{frame}	

		\def\WN[#1]{\colorbox{white!40}{#1}}
		\def\VN[#1]{\colorbox{blue!40}{#1}}
		\begin{frame}{\insertsection}
			Representações de vetores:
			\begin{itemize}
				\item Matematicamente: $v = (v_1, v_2, ..., v_{n-1}, v_n)$
				\item Computacionalmente:
					\begin{itemize}
						\item Os elementos são indexados por um índice (geralmente v[i])
						\item Geralmente começam do índice \textbf{0}
					\end{itemize}
					Exemplo:
					\begin{itemize}
						\item Vetor tamanho 4, índices de 0 a 3
					\end{itemize}
			\begin{table}
				\small
				\setlength{\tabcolsep}{0pt}	
				\begin{tabular}{|c|c|c|c|c|}
					\hline
					vetor 	&\VN[4]&\VN[6]&\VN[2]&\VN[1] \\\hline
					índice 		&\VN[0]&\VN[1]&\VN[2]&\VN[3] \\\hline
				\end{tabular}
			\end{table}
			\end{itemize}
		\end{frame}	




	\section{Vetores em C}

		\begin{frame}[c]
			\begin{center}
				\structure{\large \insertsection}
			\end{center}
		\end{frame} 

		\begin{frame}{Declarando um vetor em C} 
			Há várias opções:
			\lstinputlisting{declaracaoVetor.c}
			\begin{itemize}
				\item \textdf{único tipo}
				\item tamanho do vetor: \textdf{suficiente para caber os dados}
				\item tamanho do vetor será especificado no problema
				\item tamanhos flexíveis: (\textdf{VLA} ou \textdf{alocação dinâmica})
			\end{itemize}
		\end{frame}



		\begin{frame}{Acesso ao índice} 
			Basta identificar o elemento usando o seu \textdf{índice} entre [] (lembre-se de que começa com 0):
			\lstinputlisting{acessoVetor.c}
			\begin{table}
				\small
				\setlength{\tabcolsep}{0pt}	
				\begin{tabular}{|c|c|c|c|c|}
					\hline
					vetor 	&\VN[4]&\VN[6]&\VN[2]&\VN[1] \\\hline
					índice 		&\VN[0]&\VN[1]&\VN[2]&\VN[3] \\\hline
				\end{tabular}
			\end{table}
		\end{frame}

		\begin{frame}
			\begin{alertblock}{\ding{46} Exercício em sala}
				Declare um vetor de 5 inteiros, inicializando seus valores em 1, 4, 5, 7 e 9. Acessando o 2º número do vetor, modifique seu valor de 4 para 3. Depois escreva seus valores na tela utilizando uma estrutura de repetição.
			\end{alertblock}
		\end{frame}


		\begin{frame}
			Lembre-se de que não necessariamente precisa usar todo o vetor:
			\begin{table}
				\small
				\setlength{\tabcolsep}{0pt}	
				\begin{tabular}{|c|c|c|c|c|c|c|c|}
					\hline
					numeros &\VN[8]&\VN[7]&\VN[8]&\VN[6]&\VN[5]&\WN[8]&\WN[16] \\\hline
					índice i 		&\VN[0]&\VN[1]&\VN[2]&\VN[3]&\VN[4]&\WN[5]&\WN[6] \\\hline
					n = 5 & & & & & $\uparrow$ & & \\\hline
				\end{tabular}
			\end{table}
			\lstinputlisting{numerosDigitados.c}
		\end{frame} 


		\begin{frame}
			\begin{alertblock}{\ding{46} Exercício em sala}
				Escreva um programa em C com os seguintes passos:
				\begin{itemize}
					\item declarar um vetor de 10 inteiros
					\item ler um número inteiro \textbf{n} (assuma que o usuário digita $n \le 10$)
					\item ler \textbf{n} inteiros, armazenando-os no vetor
					\item ler um número inteiro \textbf{x}
					\item escrever na tela quantos dos \textbf{n} números são iguais a \textbf{x}
				\end{itemize}
			\end{alertblock}
		\end{frame}


		\begin{frame}
			Vetores também podem ser usados para \textbf{contar/marcar}:
			\lstinputlisting{frequencia.c}
		\end{frame} 

		\begin{frame}
			\begin{alertblock}{\ding{46} Exercício em sala}
				Escreva um programa que leia números inteiros até o usuário digitar 0. Depois, escreva na tela quantas vezes cada número de 1 a 9 foi lido do usuário.
			\end{alertblock}
		\end{frame}






		\section{Detalhes Implementacionais}
		
		\begin{frame}[c]
			\begin{center}
				\structure{\large \insertsection}
			\end{center}
		\end{frame} 

		\begin{frame}{VLA} 
		 VLA: variable length array (arranjo de tamanho variável)
			\lstinputlisting{declaracaoVetor2.c}
			\begin{itemize}
				\item \textbf{vantagem}: rápido de declarar e mais fácil de entender
				\item  \textbf{desvantagens}: 
					\begin{itemize}
						\item risco de \textdf{stack overflow}: usa memória da pilha (aprx 8MB)
						\item \textdf{fraca portabilidade}: nem todos os compiladores implementam
					\end{itemize}
				\item 	\bcattention Pelas desvantagens, \textdf{não use VLA!}, use \textdf{alocação dinâmica}
			\end{itemize}
		\end{frame}



		\begin{frame}{Falha de segmentação}
			o que acontece acessando fib[4]?
			\tiny
			\setlength{\tabcolsep}{0pt}	
			\begin{table}
				\begin{tabular}{|@{\hskip 0.2cm}c@{\hskip 0.2cm}|c|c|c|c|c|c|c|c|c|c|@{\hskip 0.2cm}c@{\hskip 0.2cm}|}
					\hline
					\textbf{Endereço} & & & & & & & & & \textbf{valor} & \textbf{tipo} & \textbf{identificação}\\\hline
					0xbffff22f & \RN[1]&\RN[1]&\RN[0]&\RN[1]&\RN[1]&\RN[1]&\RN[0]&\RN[1]& 3.2 & \textbf{real} & precoGasolina\\\hline
					0xbffff230 & \RN[1]&\RN[1]&\RN[0]&\RN[0]&\RN[1]&\RN[1]&\RN[0]&\RN[0]& & & \\\hline
					0xbffff231 & \RN[0]&\RN[1]&\RN[0]&\RN[0]&\RN[1]&\RN[1]&\RN[0]&\RN[0]& & & \\\hline
					0xbffff232 & \RN[0]&\RN[1]&\RN[0]&\RN[0]&\RN[0]&\RN[0]&\RN[0]&\RN[0]& & & \\\hline
					0xbffff233 & \BN[0]&\BN[1]&\BN[0]&\BN[0]&\BN[1]&\BN[0]&\BN[0]&\BN[1]& 1 & \textbf{inteiro curto} & fib[0]\\\hline
					0xbffff234 & \BN[0]&\BN[1]&\BN[0]&\BN[0]&\BN[1]&\BN[1]&\BN[0]&\BN[1]& 1 & \textbf{inteiro curto} & fib[1]\\\hline
					0xbffff235 & \BN[0]&\BN[1]&\BN[0]&\BN[0]&\BN[0]&\BN[1]&\BN[0]&\BN[0]& 2 & \textbf{inteiro curto} & fib[2]\\\hline
					0xbffff236 & \BN[0]&\BN[0]&\BN[1]&\BN[1]&\BN[0]&\BN[0]&\BN[0]&\BN[0]& 3 & \textbf{inteiro curto} & fib[3]\\\hline
					0xbffff237 & \GN[0]&\GN[0]&\GN[0]&\GN[0]&\GN[0]&\GN[1]&\GN[0]&\GN[1]& 5 & \textbf{inteiro curto} & valorIndice\\\hline
					0xbffff238 & \BN[0]&\BN[1]&\BN[0]&\BN[0]&\BN[0]&\BN[0]&\BN[1]&\BN[0]& B & \textbf{caractere} & letra1\\\hline
					0xbffff239 & \BN[0]&\BN[1]&\BN[0]&\BN[0]&\BN[0]&\BN[0]&\BN[1]&\BN[1]& C & \textbf{caractere} & letra2\\\hline
				\end{tabular}
			\end{table}
			\normalsize
			\bcattention \textdf{Falha de segmentação (segmentation fault)} : \textdf{acesso indevido} de memória ou a um \textdf{endereço inválido}
		\end{frame}


		\end{document}

