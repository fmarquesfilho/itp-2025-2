% Copyright 2007 by Till Tantau
%
% This file may be distributed and/or modified
%
% 1. under the LaTeX Project Public License and/or
% 2. under the GNU Public License.
%
% See the file doc/licenses/LICENSE for more details.

\documentclass[portuguese,10pt,xcolor=table]{bredelebeamer}
\setbeameroption{show notes}

\usepackage[brazil]{babel}
\usepackage[utf8]{inputenc}
\usepackage{times}
\usepackage{varwidth}
\usepackage{listings} % Código de programas
\usepackage{tikz}
\usepackage{pifont}
\usepackage[tikz]{bclogo}
\usepackage{tikzsymbols}
\usepackage{mathabx}
\let\widering\relax
\usetikzlibrary{arrows,shapes}

\usetikzlibrary{calc,decorations.pathmorphing,patterns}
\pgfdeclaredecoration{penciline}{initial}{
	\state{initial}[width=+\pgfdecoratedinputsegmentremainingdistance,
		auto corner on length=1mm,]{
			\pgfpathcurveto%
			{% From
				\pgfqpoint{\pgfdecoratedinputsegmentremainingdistance}
				{\pgfdecorationsegmentamplitude}
			}
			{%  Control 1
				\pgfmathrand
					\pgfpointadd{\pgfqpoint{\pgfdecoratedinputsegmentremainingdistance}{0pt}}
				{\pgfqpoint{-\pgfdecorationsegmentaspect
							   \pgfdecoratedinputsegmentremainingdistance}%
							   {\pgfmathresult\pgfdecorationsegmentamplitude}
				}
			}
			{%TO 
				\pgfpointadd{\pgfpointdecoratedinputsegmentlast}{\pgfpoint{1pt}{1pt}}
			}
		}
	\state{final}{}
}



\everymath{\displaystyle}
\tikzstyle{every picture}+=[remember picture,decoration=penciline]
\DeclareTextFontCommand{\textdf}{\bfseries\color{blue!80}}
%\tikzstyle{every node}+=[decorate]
%\tikzstyle{every path}+=[decorate]
%\tikzstyle{na} = [baseline=-.5ex]

\usepackage[T1]{fontenc}

\def\lecturename{IMD0012 - Introdução às técnicas de programação}

\title{\insertlecture}

\author{Prof. Fernando Figueira\\(adaptado do material do Prof. Rafael Beserra Gomes)}

\institute{UFRN}

\subject{Estruturas de Repetição Aninhadas}

\lecture[]{Estruturas de Repetição Aninhadas}{}

\date{}

\def\exe[#1]{\color{gray}#1\color{black}}
\def\exp[#1]{\color{gray}<\textit{#1}>\color{black}}
\def\espaco{\color{gray}\hspace{0.2cm}\color{black}}
\def\espaco{\color{blue}␣\color{black}}
\def\inativo[#1]{\color{gray}#1\color{black}}

\definecolor{deepgreen}{rgb}{0,0.5,0}
\definecolor{deepred}{rgb}{0.5,0,0}
\lstset{
	language=C,
	basicstyle=\footnotesize\ttfamily,
	%basicstyle=\scriptsize\ttfamily,
	%keywordstyle=\footnotesize\bfseries\sffamily,
	%keywordstyle=\scriptsize\bfseries\sffamily,
	showstringspaces=false,
	numbers=left,
	numberstyle=\footnotesize,
	stepnumber=1,
	numbersep=5pt,
	tabsize=4,
	%backgroundcolor=\color{blue!05},
	backgroundcolor=\color{gray!35},
	showspaces=false,
	showtabs=false,
	stringstyle=\ttfamily\color{red!80!brown},
	commentstyle=\ttfamily\color{blue!80},
	%keywordstyle=\bfseries\color{black},
	escapeinside={@}{@}
	}
	\renewcommand{\lstlistingname}{Código}
\begin{document}

\usebackgroundtemplate{%
	\includegraphics[width=\paperwidth,height=\paperheight]{background2}
}
\begin{frame}
  \maketitle
 \begin{center}
 \tiny
Material compilado em \today.\\
  Licença desta apresentação:\\
		\includegraphics[height=1.0cm]{by-nc-nd.png}\\
http://creativecommons.org/licenses/
	\end{center}
\end{frame}


\def\WN[#1]{\cellcolor{white!40}#1}
\def\VN[#1]{\cellcolor{blue!40}#1}
\def\UN[#1]{\cellcolor{white}#1}
\def\ZN[#1]{\cellcolor{deepgreen!90}#1}
\def\RN[#1]{\cellcolor{red!40}#1}
\def\DRN[#1]{\cellcolor{deepred!90}#1}


\section{Números primos com repetições aninhadas}

	\begin{frame}[c]
		\begin{center}
			\structure{\large \insertsection}
		\end{center}
	\end{frame} 
	
	\begin{frame}
		\begin{itemize}
			\item Escrever \textbf{todos} os números primos de 1 a 17
			\item
				\begin{table}
					\small
					\setlength{\tabcolsep}{2pt}	
					\begin{tabular}{|c| c| c| c| c| c| c| c| c| c| c| c| c| c| c| c| c| c|}
						\hline
							&\VN[1] &\VN[2] &\VN[3] &\VN[4] &\VN[5] &\VN[6] &\VN[7] &\VN[8] &\VN[9] &\WN[10] &\WN[11] &\WN[12] &\WN[13] &\WN[14] &\WN[15] &\WN[16] & \WN[17] \\\hline
						i &  &  &  &  & & & &  & $\uparrow$ &  &  &  &  &  &  & & \\\hline
					\end{tabular}
				\end{table}

				\lstinputlisting{primos.c}
		\end{itemize}
	\end{frame}	
	\begin{frame}
				\lstinputlisting{primos2.c}
	\end{frame}

	\begin{frame}
		\lstinputlisting{primos3.c}
		\only<1>{
\begin{table}
\small
\setlength{\tabcolsep}{2pt}
\begin{tabular}{|c| c| c| c| c| c| c| c| c| c| c| c| c| c| c| c| c| c| }
\hline
  &\VN[1] &\WN[2] &\WN[3] &\WN[4] &\WN[5] &\WN[6] &\WN[7] &\WN[8] &\WN[9] &\WN[10] &\WN[11] &\WN[12] &\WN[13] &\WN[14] &\WN[15] &\WN[16] &\WN[17] \\\hline \VN[i] 
&  $\uparrow$ &  &  &  &  &  &  &  &  &  &  &  &  &  &  &  &  \\\hline
  &\UN[1] & & & & & & & & & & & & & & & & \\\hline \ZN[j] 
& $\drsh$ $\dlsh$ & & & & & & & & & & & & & & & & \\\hline
\end{tabular}
\end{table}
\begin{itemize}
\item Para i = 1, ao final do for j, cont =  1
\item
\textbf{output: }[]
\end{itemize}
}
\only<2>{
\begin{table}
\small
\setlength{\tabcolsep}{2pt}
\begin{tabular}{|c| c| c| c| c| c| c| c| c| c| c| c| c| c| c| c| c| c| }
\hline
  &\WN[1] &\VN[2] &\WN[3] &\WN[4] &\WN[5] &\WN[6] &\WN[7] &\WN[8] &\WN[9] &\WN[10] &\WN[11] &\WN[12] &\WN[13] &\WN[14] &\WN[15] &\WN[16] &\WN[17] \\\hline \VN[i] 
&  &  $\uparrow$ &  &  &  &  &  &  &  &  &  &  &  &  &  &  &  \\\hline
  &\UN[1] &\UN[2] & & & & & & & & & & & & & & & \\\hline \ZN[j] 
& $\drsh$ & $\dlsh$ & & & & & & & & & & & & & & & \\\hline
\end{tabular}
\end{table}
\begin{itemize}
\item Para i = 2, ao final do for j, cont =  2
\item
\textbf{output: }[2]
\end{itemize}
}
\only<3>{
\begin{table}
\small
\setlength{\tabcolsep}{2pt}
\begin{tabular}{|c| c| c| c| c| c| c| c| c| c| c| c| c| c| c| c| c| c| }
\hline
  &\WN[1] &\WN[2] &\VN[3] &\WN[4] &\WN[5] &\WN[6] &\WN[7] &\WN[8] &\WN[9] &\WN[10] &\WN[11] &\WN[12] &\WN[13] &\WN[14] &\WN[15] &\WN[16] &\WN[17] \\\hline \VN[i] 
&  &  &  $\uparrow$ &  &  &  &  &  &  &  &  &  &  &  &  &  &  \\\hline
  &\UN[1] &\UN[2] &\UN[3] & & & & & & & & & & & & & & \\\hline \ZN[j] 
& $\drsh$ & & $\dlsh$ & & & & & & & & & & & & & & \\\hline
\end{tabular}
\end{table}
\begin{itemize}
\item Para i = 3, ao final do for j, cont =  2
\item
\textbf{output: }[2, 3]
\end{itemize}
}
\only<4>{
\begin{table}
\small
\setlength{\tabcolsep}{2pt}
\begin{tabular}{|c| c| c| c| c| c| c| c| c| c| c| c| c| c| c| c| c| c| }
\hline
  &\WN[1] &\WN[2] &\WN[3] &\VN[4] &\WN[5] &\WN[6] &\WN[7] &\WN[8] &\WN[9] &\WN[10] &\WN[11] &\WN[12] &\WN[13] &\WN[14] &\WN[15] &\WN[16] &\WN[17] \\\hline \VN[i] 
&  &  &  &  $\uparrow$ &  &  &  &  &  &  &  &  &  &  &  &  &  \\\hline
  &\UN[1] &\UN[2] &\UN[3] &\UN[4] & & & & & & & & & & & & & \\\hline \ZN[j] 
& $\drsh$ & & & $\dlsh$ & & & & & & & & & & & & & \\\hline
\end{tabular}
\end{table}
\begin{itemize}
\item Para i = 4, ao final do for j, cont =  3
\item
\textbf{output: }[2, 3]
\end{itemize}
}
\only<5>{
\begin{table}
\small
\setlength{\tabcolsep}{2pt}
\begin{tabular}{|c| c| c| c| c| c| c| c| c| c| c| c| c| c| c| c| c| c| }
\hline
  &\WN[1] &\WN[2] &\WN[3] &\WN[4] &\VN[5] &\WN[6] &\WN[7] &\WN[8] &\WN[9] &\WN[10] &\WN[11] &\WN[12] &\WN[13] &\WN[14] &\WN[15] &\WN[16] &\WN[17] \\\hline \VN[i] 
&  &  &  &  &  $\uparrow$ &  &  &  &  &  &  &  &  &  &  &  &  \\\hline
  &\UN[1] &\UN[2] &\UN[3] &\UN[4] &\UN[5] & & & & & & & & & & & & \\\hline \ZN[j] 
& $\drsh$ & & & & $\dlsh$ & & & & & & & & & & & & \\\hline
\end{tabular}
\end{table}
\begin{itemize}
\item Para i = 5, ao final do for j, cont =  2
\item
\textbf{output: }[2, 3, 5]
\end{itemize}
}
\only<6>{
\begin{table}
\small
\setlength{\tabcolsep}{2pt}
\begin{tabular}{|c| c| c| c| c| c| c| c| c| c| c| c| c| c| c| c| c| c| }
\hline
  &\WN[1] &\WN[2] &\WN[3] &\WN[4] &\WN[5] &\VN[6] &\WN[7] &\WN[8] &\WN[9] &\WN[10] &\WN[11] &\WN[12] &\WN[13] &\WN[14] &\WN[15] &\WN[16] &\WN[17] \\\hline \VN[i] 
&  &  &  &  &  &  $\uparrow$ &  &  &  &  &  &  &  &  &  &  &  \\\hline
  &\UN[1] &\UN[2] &\UN[3] &\UN[4] &\UN[5] &\UN[6] & & & & & & & & & & & \\\hline \ZN[j] 
& $\drsh$ & & & & & $\dlsh$ & & & & & & & & & & & \\\hline
\end{tabular}
\end{table}
\begin{itemize}
\item Para i = 6, ao final do for j, cont =  4
\item
\textbf{output: }[2, 3, 5]
\end{itemize}
}
\only<7>{
\begin{table}
\small
\setlength{\tabcolsep}{2pt}
\begin{tabular}{|c| c| c| c| c| c| c| c| c| c| c| c| c| c| c| c| c| c| }
\hline
  &\WN[1] &\WN[2] &\WN[3] &\WN[4] &\WN[5] &\WN[6] &\VN[7] &\WN[8] &\WN[9] &\WN[10] &\WN[11] &\WN[12] &\WN[13] &\WN[14] &\WN[15] &\WN[16] &\WN[17] \\\hline \VN[i] 
&  &  &  &  &  &  &  $\uparrow$ &  &  &  &  &  &  &  &  &  &  \\\hline
  &\UN[1] &\UN[2] &\UN[3] &\UN[4] &\UN[5] &\UN[6] &\UN[7] & & & & & & & & & & \\\hline \ZN[j] 
& $\drsh$ & & & & & & $\dlsh$ & & & & & & & & & & \\\hline
\end{tabular}
\end{table}
\begin{itemize}
\item Para i = 7, ao final do for j, cont =  2
\item
\textbf{output: }[2, 3, 5, 7]
\end{itemize}
}
\only<8>{
\begin{table}
\small
\setlength{\tabcolsep}{2pt}
\begin{tabular}{|c| c| c| c| c| c| c| c| c| c| c| c| c| c| c| c| c| c| }
\hline
  &\WN[1] &\WN[2] &\WN[3] &\WN[4] &\WN[5] &\WN[6] &\WN[7] &\VN[8] &\WN[9] &\WN[10] &\WN[11] &\WN[12] &\WN[13] &\WN[14] &\WN[15] &\WN[16] &\WN[17] \\\hline \VN[i] 
&  &  &  &  &  &  &  &  $\uparrow$ &  &  &  &  &  &  &  &  &  \\\hline
  &\UN[1] &\UN[2] &\UN[3] &\UN[4] &\UN[5] &\UN[6] &\UN[7] &\UN[8] & & & & & & & & & \\\hline \ZN[j] 
& $\drsh$ & & & & & & & $\dlsh$ & & & & & & & & & \\\hline
\end{tabular}
\end{table}
\begin{itemize}
\item Para i = 8, ao final do for j, cont =  4
\item
\textbf{output: }[2, 3, 5, 7]
\end{itemize}
}
\only<9>{
\begin{table}
\small
\setlength{\tabcolsep}{2pt}
\begin{tabular}{|c| c| c| c| c| c| c| c| c| c| c| c| c| c| c| c| c| c| }
\hline
  &\WN[1] &\WN[2] &\WN[3] &\WN[4] &\WN[5] &\WN[6] &\WN[7] &\WN[8] &\VN[9] &\WN[10] &\WN[11] &\WN[12] &\WN[13] &\WN[14] &\WN[15] &\WN[16] &\WN[17] \\\hline \VN[i] 
&  &  &  &  &  &  &  &  &  $\uparrow$ &  &  &  &  &  &  &  &  \\\hline
  &\UN[1] &\UN[2] &\UN[3] &\UN[4] &\UN[5] &\UN[6] &\UN[7] &\UN[8] &\UN[9] & & & & & & & & \\\hline \ZN[j] 
& $\drsh$ & & & & & & & & $\dlsh$ & & & & & & & & \\\hline
\end{tabular}
\end{table}
\begin{itemize}
\item Para i = 9, ao final do for j, cont =  3
\item
\textbf{output: }[2, 3, 5, 7]
\end{itemize}
}
\only<10>{
\begin{table}
\small
\setlength{\tabcolsep}{2pt}
\begin{tabular}{|c| c| c| c| c| c| c| c| c| c| c| c| c| c| c| c| c| c| }
\hline
  &\WN[1] &\WN[2] &\WN[3] &\WN[4] &\WN[5] &\WN[6] &\WN[7] &\WN[8] &\WN[9] &\VN[10] &\WN[11] &\WN[12] &\WN[13] &\WN[14] &\WN[15] &\WN[16] &\WN[17] \\\hline \VN[i] 
&  &  &  &  &  &  &  &  &  &  $\uparrow$ &  &  &  &  &  &  &  \\\hline
  &\UN[1] &\UN[2] &\UN[3] &\UN[4] &\UN[5] &\UN[6] &\UN[7] &\UN[8] &\UN[9] &\UN[10] & & & & & & & \\\hline \ZN[j] 
& $\drsh$ & & & & & & & & & $\dlsh$ & & & & & & & \\\hline
\end{tabular}
\end{table}
\begin{itemize}
\item Para i = 10, ao final do for j, cont =  4
\item
\textbf{output: }[2, 3, 5, 7]
\end{itemize}
}
\only<11>{
\begin{table}
\small
\setlength{\tabcolsep}{2pt}
\begin{tabular}{|c| c| c| c| c| c| c| c| c| c| c| c| c| c| c| c| c| c| }
\hline
  &\WN[1] &\WN[2] &\WN[3] &\WN[4] &\WN[5] &\WN[6] &\WN[7] &\WN[8] &\WN[9] &\WN[10] &\VN[11] &\WN[12] &\WN[13] &\WN[14] &\WN[15] &\WN[16] &\WN[17] \\\hline \VN[i] 
&  &  &  &  &  &  &  &  &  &  &  $\uparrow$ &  &  &  &  &  &  \\\hline
  &\UN[1] &\UN[2] &\UN[3] &\UN[4] &\UN[5] &\UN[6] &\UN[7] &\UN[8] &\UN[9] &\UN[10] &\UN[11] & & & & & & \\\hline \ZN[j] 
& $\drsh$ & & & & & & & & & & $\dlsh$ & & & & & & \\\hline
\end{tabular}
\end{table}
\begin{itemize}
\item Para i = 11, ao final do for j, cont =  2
\item
\textbf{output: }[2, 3, 5, 7, 11]
\end{itemize}
}
\only<12>{
\begin{table}
\small
\setlength{\tabcolsep}{2pt}
\begin{tabular}{|c| c| c| c| c| c| c| c| c| c| c| c| c| c| c| c| c| c| }
\hline
  &\WN[1] &\WN[2] &\WN[3] &\WN[4] &\WN[5] &\WN[6] &\WN[7] &\WN[8] &\WN[9] &\WN[10] &\WN[11] &\VN[12] &\WN[13] &\WN[14] &\WN[15] &\WN[16] &\WN[17] \\\hline \VN[i] 
&  &  &  &  &  &  &  &  &  &  &  &  $\uparrow$ &  &  &  &  &  \\\hline
  &\UN[1] &\UN[2] &\UN[3] &\UN[4] &\UN[5] &\UN[6] &\UN[7] &\UN[8] &\UN[9] &\UN[10] &\UN[11] &\UN[12] & & & & & \\\hline \ZN[j] 
& $\drsh$ & & & & & & & & & & & $\dlsh$ & & & & & \\\hline
\end{tabular}
\end{table}
\begin{itemize}
\item Para i = 12, ao final do for j, cont =  6
\item
\textbf{output: }[2, 3, 5, 7, 11]
\end{itemize}
}
\only<13>{
\begin{table}
\small
\setlength{\tabcolsep}{2pt}
\begin{tabular}{|c| c| c| c| c| c| c| c| c| c| c| c| c| c| c| c| c| c| }
\hline
  &\WN[1] &\WN[2] &\WN[3] &\WN[4] &\WN[5] &\WN[6] &\WN[7] &\WN[8] &\WN[9] &\WN[10] &\WN[11] &\WN[12] &\VN[13] &\WN[14] &\WN[15] &\WN[16] &\WN[17] \\\hline \VN[i] 
&  &  &  &  &  &  &  &  &  &  &  &  &  $\uparrow$ &  &  &  &  \\\hline
  &\UN[1] &\UN[2] &\UN[3] &\UN[4] &\UN[5] &\UN[6] &\UN[7] &\UN[8] &\UN[9] &\UN[10] &\UN[11] &\UN[12] &\UN[13] & & & & \\\hline \ZN[j] 
& $\drsh$ & & & & & & & & & & & & $\dlsh$ & & & & \\\hline
\end{tabular}
\end{table}
\begin{itemize}
\item Para i = 13, ao final do for j, cont =  2
\item
\textbf{output: }[2, 3, 5, 7, 11, 13]
\end{itemize}
}
\only<14>{
\begin{table}
\small
\setlength{\tabcolsep}{2pt}
\begin{tabular}{|c| c| c| c| c| c| c| c| c| c| c| c| c| c| c| c| c| c| }
\hline
  &\WN[1] &\WN[2] &\WN[3] &\WN[4] &\WN[5] &\WN[6] &\WN[7] &\WN[8] &\WN[9] &\WN[10] &\WN[11] &\WN[12] &\WN[13] &\VN[14] &\WN[15] &\WN[16] &\WN[17] \\\hline \VN[i] 
&  &  &  &  &  &  &  &  &  &  &  &  &  &  $\uparrow$ &  &  &  \\\hline
  &\UN[1] &\UN[2] &\UN[3] &\UN[4] &\UN[5] &\UN[6] &\UN[7] &\UN[8] &\UN[9] &\UN[10] &\UN[11] &\UN[12] &\UN[13] &\UN[14] & & & \\\hline \ZN[j] 
& $\drsh$ & & & & & & & & & & & & & $\dlsh$ & & & \\\hline
\end{tabular}
\end{table}
\begin{itemize}
\item Para i = 14, ao final do for j, cont =  4
\item
\textbf{output: }[2, 3, 5, 7, 11, 13]
\end{itemize}
}
\only<15>{
\begin{table}
\small
\setlength{\tabcolsep}{2pt}
\begin{tabular}{|c| c| c| c| c| c| c| c| c| c| c| c| c| c| c| c| c| c| }
\hline
  &\WN[1] &\WN[2] &\WN[3] &\WN[4] &\WN[5] &\WN[6] &\WN[7] &\WN[8] &\WN[9] &\WN[10] &\WN[11] &\WN[12] &\WN[13] &\WN[14] &\VN[15] &\WN[16] &\WN[17] \\\hline \VN[i] 
&  &  &  &  &  &  &  &  &  &  &  &  &  &  &  $\uparrow$ &  &  \\\hline
  &\UN[1] &\UN[2] &\UN[3] &\UN[4] &\UN[5] &\UN[6] &\UN[7] &\UN[8] &\UN[9] &\UN[10] &\UN[11] &\UN[12] &\UN[13] &\UN[14] &\UN[15] & & \\\hline \ZN[j] 
& $\drsh$ & & & & & & & & & & & & & & $\dlsh$ & & \\\hline
\end{tabular}
\end{table}
\begin{itemize}
\item Para i = 15, ao final do for j, cont =  4
\item
\textbf{output: }[2, 3, 5, 7, 11, 13]
\end{itemize}
}
\only<16>{
\begin{table}
\small
\setlength{\tabcolsep}{2pt}
\begin{tabular}{|c| c| c| c| c| c| c| c| c| c| c| c| c| c| c| c| c| c| }
\hline
  &\WN[1] &\WN[2] &\WN[3] &\WN[4] &\WN[5] &\WN[6] &\WN[7] &\WN[8] &\WN[9] &\WN[10] &\WN[11] &\WN[12] &\WN[13] &\WN[14] &\WN[15] &\VN[16] &\WN[17] \\\hline \VN[i] 
&  &  &  &  &  &  &  &  &  &  &  &  &  &  &  &  $\uparrow$ &  \\\hline
  &\UN[1] &\UN[2] &\UN[3] &\UN[4] &\UN[5] &\UN[6] &\UN[7] &\UN[8] &\UN[9] &\UN[10] &\UN[11] &\UN[12] &\UN[13] &\UN[14] &\UN[15] &\UN[16] & \\\hline \ZN[j] 
& $\drsh$ & & & & & & & & & & & & & & & $\dlsh$ & \\\hline
\end{tabular}
\end{table}
\begin{itemize}
\item Para i = 16, ao final do for j, cont =  5
\item
\textbf{output: }[2, 3, 5, 7, 11, 13]
\end{itemize}
}
\only<17>{
\begin{table}
\small
\setlength{\tabcolsep}{2pt}
\begin{tabular}{|c| c| c| c| c| c| c| c| c| c| c| c| c| c| c| c| c| c| }
\hline
  &\WN[1] &\WN[2] &\WN[3] &\WN[4] &\WN[5] &\WN[6] &\WN[7] &\WN[8] &\WN[9] &\WN[10] &\WN[11] &\WN[12] &\WN[13] &\WN[14] &\WN[15] &\WN[16] &\VN[17] \\\hline \VN[i] 
&  &  &  &  &  &  &  &  &  &  &  &  &  &  &  &  &  $\uparrow$ \\\hline
  &\UN[1] &\UN[2] &\UN[3] &\UN[4] &\UN[5] &\UN[6] &\UN[7] &\UN[8] &\UN[9] &\UN[10] &\UN[11] &\UN[12] &\UN[13] &\UN[14] &\UN[15] &\UN[16] &\UN[17] \\\hline \ZN[j] 
& $\drsh$ & & & & & & & & & & & & & & & & $\dlsh$ \\\hline
\end{tabular}
\end{table}
\begin{itemize}
\item Para i = 17, ao final do for j, cont =  2
\item
\textbf{output: }[2, 3, 5, 7, 11, 13, 17]
\end{itemize}
}


	\end{frame}


\end{document}

