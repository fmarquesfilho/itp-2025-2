% Copyright 2007 by Till Tantau
%
% This file may be distributed and/or modified
%
% 1. under the LaTeX Project Public License and/or
% 2. under the GNU Public License.
%
% See the file doc/licenses/LICENSE for more details.

\documentclass[portuguese,10pt,xcolor=table]{bredelebeamer}
\setbeameroption{show notes}

\usepackage[brazil]{babel}
\usepackage[utf8]{inputenc}
\usepackage{times}
\usepackage{varwidth}
\usepackage{listings} % Código de programas
\usepackage{tikz}
\usepackage{pifont}
\usepackage[tikz]{bclogo}
\usepackage{tikzsymbols}
\usepackage{fourier}
\usepackage{mathabx}
\let\widering\relax
\usetikzlibrary{arrows,shapes}

\usetikzlibrary{calc,decorations.pathmorphing,patterns}
\pgfdeclaredecoration{penciline}{initial}{
	\state{initial}[width=+\pgfdecoratedinputsegmentremainingdistance,
		auto corner on length=1mm,]{
			\pgfpathcurveto%
			{% From
				\pgfqpoint{\pgfdecoratedinputsegmentremainingdistance}
				{\pgfdecorationsegmentamplitude}
			}
			{%  Control 1
				\pgfmathrand
					\pgfpointadd{\pgfqpoint{\pgfdecoratedinputsegmentremainingdistance}{0pt}}
				{\pgfqpoint{-\pgfdecorationsegmentaspect
							   \pgfdecoratedinputsegmentremainingdistance}%
							   {\pgfmathresult\pgfdecorationsegmentamplitude}
				}
			}
			{%TO 
				\pgfpointadd{\pgfpointdecoratedinputsegmentlast}{\pgfpoint{1pt}{1pt}}
			}
		}
	\state{final}{}
}



\everymath{\displaystyle}
\tikzstyle{every picture}+=[remember picture,decoration=penciline]
\DeclareTextFontCommand{\textdf}{\bfseries\color{blue!80}}
%\tikzstyle{every node}+=[decorate]
%\tikzstyle{every path}+=[decorate]
%\tikzstyle{na} = [baseline=-.5ex]

\usepackage[T1]{fontenc}

\def\lecturename{IMD0012 - Introdução às técnicas de programação}

\title{\insertlecture}

\author{Prof. Fernando Figueira\\(adaptado do material do Prof. Rafael Beserra Gomes)}

\institute{UFRN}

\subject{Estruturas de Repetição Aninhadas}

\lecture[]{Estruturas de Repetição Aninhadas}{}

\date{}

\def\exe[#1]{\color{gray}#1\color{black}}
\def\exp[#1]{\color{gray}<\textit{#1}>\color{black}}
\def\espaco{\color{gray}\hspace{0.2cm}\color{black}}
\def\espaco{\color{blue}␣\color{black}}
\def\inativo[#1]{\color{gray}#1\color{black}}

\definecolor{deepgreen}{rgb}{0,0.5,0}
\definecolor{deepred}{rgb}{0.5,0,0}
\lstset{
	language=C,
	basicstyle=\footnotesize\ttfamily,
	%basicstyle=\scriptsize\ttfamily,
	keywordstyle=\footnotesize\bfseries\sffamily,
	%keywordstyle=\scriptsize\bfseries\sffamily,
	showstringspaces=false,
	numbers=left,
	numberstyle=\footnotesize,
	stepnumber=1,
	numbersep=5pt,
	tabsize=4,
	%backgroundcolor=\color{blue!05},
	backgroundcolor=\color{gray!35},
	showspaces=false,
	showtabs=false,
	stringstyle=\ttfamily\color{red!80!brown},
	commentstyle=\ttfamily\color{blue!80},
	keywordstyle=\bfseries\color{deepgreen},
	escapeinside={@}{@}
	}
	\renewcommand{\lstlistingname}{Código}
\begin{document}

\usebackgroundtemplate{%
	\includegraphics[width=\paperwidth,height=\paperheight]{background2}
}
\begin{frame}
  \maketitle
 \begin{center}
 \tiny
Material compilado em \today.\\
  Licença desta apresentação:\\
		\includegraphics[height=1.0cm]{by-nc-nd.png}\\
http://creativecommons.org/licenses/
	\end{center}
\end{frame}


\def\WN[#1]{\cellcolor{white!40}#1}
\def\VN[#1]{\cellcolor{blue!40}#1}
\def\UN[#1]{\cellcolor{green!40}#1}
\def\ZN[#1]{\cellcolor{deepgreen!90}#1}
\def\RN[#1]{\cellcolor{red!40}#1}
\def\DRN[#1]{\cellcolor{deepred!90}#1}


\section{Estruturas de repetição aninhadas}

	\begin{frame}[c]
		\begin{center}
			\structure{\large \insertsection}
		\end{center}
	\end{frame} 
	

	\begin{frame}
	\begin{alertblock}{\ding{46} Exercício em sala}
		Usando \textbf{estruturas de repetição aninhadas} escreva um programa em C para escrever 5 vezes na tela os números de 1 a 9\\
		\colorbox{gray!15}{
			\begin{varwidth}{700px}
				1 2 3 4 5 6 7 8 9\\
				1 2 3 4 5 6 7 8 9\\
				1 2 3 4 5 6 7 8 9\\
				1 2 3 4 5 6 7 8 9\\
				1 2 3 4 5 6 7 8 9\\
			\end{varwidth}
		}\\
	\end{alertblock}
	\end{frame}

	\begin{frame}
	\begin{alertblock}{\ding{46} Exercício em sala}
		Usando \textbf{estruturas de repetição aninhadas} escreva um programa em C para escrever na tela o seguinte padrão:\\
		\colorbox{gray!15}{
			\begin{varwidth}{700px}
				1\\
				1 2\\
				1 2 3\\
				1 2 3 4\\
				1 2 3 4 5\\
				1 2 3 4 5 6\\
				1 2 3 4 5 6 7\\
				1 2 3 4 5 6 7 8\\
				1 2 3 4 5 6 7 8 9
			\end{varwidth}
		}\\
	\end{alertblock}
	\end{frame}





\end{document}

