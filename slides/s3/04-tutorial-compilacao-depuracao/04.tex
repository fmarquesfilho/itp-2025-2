\documentclass[aspectratio=169]{beamer}
\usepackage[utf8]{inputenc}
\usepackage[brazilian]{babel}
\usepackage{listings}
\usepackage{xcolor}
\usepackage{graphicx}
\usepackage{hyperref}
\usepackage{bookmark}

% Tema
\usetheme{Madrid}
\usecolortheme{default}

% Configuração do listings para código C
\lstset{
    language=C,
    basicstyle=\ttfamily\footnotesize,
    keywordstyle=\color{blue}\bfseries,
    commentstyle=\color{gray},
    stringstyle=\color{red},
    numbers=left,
    numberstyle=\tiny\color{gray},
    stepnumber=1,
    numbersep=5pt,
    backgroundcolor=\color{lightgray!10},
    showspaces=false,
    showstringspaces=false,
    showtabs=false,
    frame=single,
    rulecolor=\color{black},
    tabsize=2,
    captionpos=b,
    breaklines=true,
    breakatwhitespace=false,
    escapeinside={\%*}{*)},
    morekeywords={scanf, printf}
}

% Definir linguagem JSON para o listings
\lstdefinelanguage{json}{
    basicstyle=\ttfamily\tiny,
    numbers=left,
    numberstyle=\tiny\color{gray},
    stepnumber=1,
    numbersep=5pt,
    showstringspaces=false,
    breaklines=true,
    frame=single,
    rulecolor=\color{black},
    morestring=[b]",
    morecomment=[l]{//},
    morecomment=[s]{/*}{*/},
    commentstyle=\color{gray},
    stringstyle=\color{red},
    literate=
     *{0}{{{\color{blue}0}}}{1}
      {1}{{{\color{blue}1}}}{1}
      {2}{{{\color{blue}2}}}{1}
      {3}{{{\color{blue}3}}}{1}
      {4}{{{\color{blue}4}}}{1}
      {5}{{{\color{blue}5}}}{1}
      {6}{{{\color{blue}6}}}{1}
      {7}{{{\color{blue}7}}}{1}
      {8}{{{\color{blue}8}}}{1}
      {9}{{{\color{blue}9}}}{1}
      {:}{{{\color{purple}{:}}}}{1}
      {,}{{{\color{purple}{,}}}}{1}
      {\{}{{{\color{orange}{\{}}}}{1}
      {\}}{{{\color{orange}{\}}}}}{1}
      {[}{{{\color{orange}{[}}}}{1}
      {]}{{{\color{orange}{]}}}}{1},
}

\title{Compilação e Depuração com GCC/GDB e VS Code}
\subtitle{Introdução às Técnicas de Programação}
\author{Prof. Fernando Figueira}
\date{Setembro 2025}

\begin{document}

\frame{\titlepage}

\begin{frame}
\frametitle{Agenda}
\tableofcontents
\end{frame}

\section{Revisão: Processo de Compilação}

\begin{frame}[fragile]
\frametitle{O que é Compilação?}
\begin{itemize}
    \item \textbf{Compilação}: Processo de transformar código-fonte em código executável
    \item \textbf{GCC}: GNU Compiler Collection - compilador gratuito e amplamente usado
    \item \textbf{Etapas}:
    \begin{enumerate}
        \item Pré-processamento (diretivas \texttt{\#include}, \texttt{\#define})
        \item Compilação (código C → assembly)
        \item Montagem (assembly → código objeto)
        \item Ligação (código objeto → executável)
    \end{enumerate}
\end{itemize}

\begin{lstlisting}[caption=Comando básico de compilação]
gcc arquivo.c -o programa
\end{lstlisting}
\end{frame}

\begin{frame}[fragile]
\frametitle{Compilação Básica}
\begin{block}{Comandos essenciais}
\begin{lstlisting}[language=bash]
# Compilacao simples
gcc divide_outro.c -o divide_outro

# Execucao
./divide_outro

# Compilacao com avisos (recomendado)
gcc -Wall divide_outro.c -o divide_outro

# Compilacao para debug
gcc -g -Wall divide_outro.c -o divide_outro
\end{lstlisting}
\end{block}

\begin{itemize}
    \item \texttt{-Wall}: Habilita avisos do compilador
    \item \texttt{-g}: Inclui informações de debug no executável
    \item \texttt{-o}: Especifica o nome do arquivo de saída
\end{itemize}
\end{frame}

\section{Depuração com GDB}

\begin{frame}
\frametitle{O que é Depuração?}
\begin{itemize}
    \item \textbf{Depuração}: Processo de encontrar e corrigir erros (bugs) no código
    \item \textbf{GDB}: GNU Debugger - ferramenta de linha de comando para depuração
    \item \textbf{Quando usar?}:
    \begin{itemize}
        \item Programa trava ou gera erro
        \item Resultado incorreto
        \item Entender fluxo de execução
        \item Verificar valores de variáveis durante execução
    \end{itemize}
\end{itemize}

\begin{block}{Pré-requisito}
Compilar com flag \texttt{-g} para incluir informações de debug
\end{block}
\end{frame}

\begin{frame}[fragile]
\frametitle{GDB: Comandos Básicos}
\begin{block}{Iniciando o GDB}
\begin{lstlisting}[language=bash]
# Compilar com debug
gcc -g -Wall divide_outro.c -o divide_outro

# Iniciar GDB
gdb ./divide_outro
\end{lstlisting}
\end{block}

\begin{columns}
\begin{column}{0.5\textwidth}
\textbf{Comandos principais:}
\begin{itemize}
    \item \texttt{run} (r) - Executar
    \item \texttt{break} (b) - Breakpoint
    \item \texttt{step} (s) - Próxima linha
    \item \texttt{next} (n) - Próxima linha (sem entrar em funções)
    \item \texttt{continue} (c) - Continuar
\end{itemize}
\end{column}
\begin{column}{0.5\textwidth}
\textbf{Inspeção:}
\begin{itemize}
    \item \texttt{print} (p) - Valor de variável
    \item \texttt{info locals} - Todas as variáveis locais
    \item \texttt{list} (l) - Ver código
    \item \texttt{quit} (q) - Sair
\end{itemize}
\end{column}
\end{columns}
\end{frame}

\begin{frame}[fragile]
\frametitle{GDB: Exemplo Prático}
\begin{block}{Sessão de debug paso-a-passo}
\begin{lstlisting}[language=bash]
$ gdb ./divide_outro
(gdb) break main          # Breakpoint na funcao main
(gdb) run                 # Executar programa
(gdb) list                # Ver codigo atual
(gdb) break 11            # Breakpoint na linha 11
(gdb) continue            # Continuar ate breakpoint
12 24                     # Entrada: digitar dois numeros
(gdb) print a             # Ver valor de 'a'
(gdb) print b             # Ver valor de 'b'
(gdb) step               # Executar proxima linha
(gdb) print resto        # Ver valor de 'resto'
(gdb) continue           # Continuar execucao
\end{lstlisting}
\end{block}
\end{frame}

\section{Depuração no VS Code}

\begin{frame}
\frametitle{VS Code: Configuração Inicial}
\begin{itemize}
    \item \textbf{Extensões necessárias}:
    \begin{itemize}
        \item C/C++ (Microsoft)
        \item C/C++ Extension Pack (opcional, mas recomendado)
    \end{itemize}
    \item \textbf{Arquivos de configuração}:
    \begin{itemize}
        \item \texttt{.vscode/tasks.json} - Tarefas de compilação
        \item \texttt{.vscode/launch.json} - Configuração de debug
    \end{itemize}
\end{itemize}

\begin{block}{Dica}
VS Code pode gerar essas configurações automaticamente quando você tenta debugar pela primeira vez
\end{block}
\end{frame}

\begin{frame}
\frametitle{VS Code: Usando o Debugger}
\begin{enumerate}
    \item \textbf{Definir breakpoints}: Clicar na margem esquerda do editor (números das linhas)
    \item \textbf{Iniciar debug}: F5 ou menu Debug → Start Debugging
    \item \textbf{Controles durante debug}:
    \begin{itemize}
        \item F10 - Step Over (próxima linha)
        \item F11 - Step Into (entrar em função)
        \item Shift+F11 - Step Out (sair de função)
        \item F5 - Continue
        \item Shift+F5 - Stop
    \end{itemize}
    \item \textbf{Painéis importantes}:
    \begin{itemize}
        \item \textbf{Variables} - Valores das variáveis
        \item \textbf{Watch} - Expressões para monitorar
        \item \textbf{Call Stack} - Pilha de chamadas
        \item \textbf{Terminal} - Onde digitar entrada do programa
    \end{itemize}
\end{enumerate}
\end{frame}

\begin{frame}[fragile]
\frametitle{IMPORTANTE: Entrada de Dados no VS Code}
\begin{alertblock}{Onde digitar quando o programa pede entrada?}
\begin{itemize}
    \item \textbf{Durante debug}: Digite no painel \textbf{TERMINAL} (não no Debug Console)
    \item \textbf{Localização}: Parte inferior da tela, aba "TERMINAL"
    \item \textbf{Exemplo}: Quando \texttt{scanf("\%d \%d", \&a, \&b)} aguarda entrada
\end{itemize}
\end{alertblock}

\begin{block}{Fluxo típico}
\begin{enumerate}
    \item Programa executa até \texttt{scanf}
    \item Execução pausa aguardando entrada
    \item Clique na aba \texttt{TERMINAL}
    \item Digite os valores (ex: \texttt{12 24})
    \item Pressione Enter
    \item Programa continua
\end{enumerate}
\end{block}
\end{frame}

\section{Demonstração Prática}

\begin{frame}
\frametitle{Vamos Praticar!}
\begin{block}{Cenário: Debugar divide\_outro.c}
Vamos encontrar e entender o comportamento do programa:
\begin{enumerate}
    \item Compilar com GCC
    \item Debugar via linha de comando (GDB)
    \item Debugar no VS Code
    \item Comparar as duas abordagens
\end{enumerate}
\end{block}

\begin{block}{Casos de teste}
\begin{itemize}
    \item Entrada: 12 4 (4 divide 12)
    \item Entrada: 7 3 (nenhum divide o outro)
    \item Entrada: 15 15 (números iguais)
\end{itemize}
\end{block}
\end{frame}

\begin{frame}
\frametitle{Exercício para Casa}
\begin{block}{Prática individual}
\begin{enumerate}
    \item Crie um programa em C que calcule o fatorial de um número
    \item Inclua pelo menos um erro intencional (ex: condição de parada incorreta)
    \item Use GDB para encontrar e corrigir o erro
    \item Repita o processo no VS Code
    \item Documente as diferenças entre as duas abordagens
\end{enumerate}
\end{block}

\begin{block}{Entrega}
Não é necessário entregar - é para fixação do conteúdo
\end{block}
\end{frame}

\begin{frame}
\frametitle{Resumo}
\begin{itemize}
    \item \textbf{GCC}: Compilador essencial para C
    \item \textbf{Flag -g}: Fundamental para debug
    \item \textbf{GDB}: Poderoso debugger de linha de comando
    \item \textbf{VS Code}: Interface visual amigável para debug
    \item \textbf{Breakpoints}: Ferramenta essencial para inspeção
    \item \textbf{Entrada no VS Code}: Use o painel TERMINAL durante debug
\end{itemize}

\begin{block}{Próxima aula}
Estruturas de repetição (for, while, do-while)
\end{block}
\end{frame}

\begin{frame}
\frametitle{Perguntas?}
\begin{center}
{\Large Dúvidas sobre compilação e depuração?}

\vspace{1cm}

{\large Obrigado!}
\end{center}
\end{frame}

\end{document}